% Options for packages loaded elsewhere
\PassOptionsToPackage{unicode}{hyperref}
\PassOptionsToPackage{hyphens}{url}
%
\documentclass[
]{book}
\usepackage{lmodern}
\usepackage{amsmath}
\usepackage{ifxetex,ifluatex}
\ifnum 0\ifxetex 1\fi\ifluatex 1\fi=0 % if pdftex
  \usepackage[T1]{fontenc}
  \usepackage[utf8]{inputenc}
  \usepackage{textcomp} % provide euro and other symbols
  \usepackage{amssymb}
\else % if luatex or xetex
  \usepackage{unicode-math}
  \defaultfontfeatures{Scale=MatchLowercase}
  \defaultfontfeatures[\rmfamily]{Ligatures=TeX,Scale=1}
\fi
% Use upquote if available, for straight quotes in verbatim environments
\IfFileExists{upquote.sty}{\usepackage{upquote}}{}
\IfFileExists{microtype.sty}{% use microtype if available
  \usepackage[]{microtype}
  \UseMicrotypeSet[protrusion]{basicmath} % disable protrusion for tt fonts
}{}
\makeatletter
\@ifundefined{KOMAClassName}{% if non-KOMA class
  \IfFileExists{parskip.sty}{%
    \usepackage{parskip}
  }{% else
    \setlength{\parindent}{0pt}
    \setlength{\parskip}{6pt plus 2pt minus 1pt}}
}{% if KOMA class
  \KOMAoptions{parskip=half}}
\makeatother
\usepackage{xcolor}
\IfFileExists{xurl.sty}{\usepackage{xurl}}{} % add URL line breaks if available
\IfFileExists{bookmark.sty}{\usepackage{bookmark}}{\usepackage{hyperref}}
\hypersetup{
  pdftitle={Communication Theories},
  pdfauthor={Mike Nguyen},
  hidelinks,
  pdfcreator={LaTeX via pandoc}}
\urlstyle{same} % disable monospaced font for URLs
\usepackage{color}
\usepackage{fancyvrb}
\newcommand{\VerbBar}{|}
\newcommand{\VERB}{\Verb[commandchars=\\\{\}]}
\DefineVerbatimEnvironment{Highlighting}{Verbatim}{commandchars=\\\{\}}
% Add ',fontsize=\small' for more characters per line
\usepackage{framed}
\definecolor{shadecolor}{RGB}{248,248,248}
\newenvironment{Shaded}{\begin{snugshade}}{\end{snugshade}}
\newcommand{\AlertTok}[1]{\textcolor[rgb]{0.94,0.16,0.16}{#1}}
\newcommand{\AnnotationTok}[1]{\textcolor[rgb]{0.56,0.35,0.01}{\textbf{\textit{#1}}}}
\newcommand{\AttributeTok}[1]{\textcolor[rgb]{0.77,0.63,0.00}{#1}}
\newcommand{\BaseNTok}[1]{\textcolor[rgb]{0.00,0.00,0.81}{#1}}
\newcommand{\BuiltInTok}[1]{#1}
\newcommand{\CharTok}[1]{\textcolor[rgb]{0.31,0.60,0.02}{#1}}
\newcommand{\CommentTok}[1]{\textcolor[rgb]{0.56,0.35,0.01}{\textit{#1}}}
\newcommand{\CommentVarTok}[1]{\textcolor[rgb]{0.56,0.35,0.01}{\textbf{\textit{#1}}}}
\newcommand{\ConstantTok}[1]{\textcolor[rgb]{0.00,0.00,0.00}{#1}}
\newcommand{\ControlFlowTok}[1]{\textcolor[rgb]{0.13,0.29,0.53}{\textbf{#1}}}
\newcommand{\DataTypeTok}[1]{\textcolor[rgb]{0.13,0.29,0.53}{#1}}
\newcommand{\DecValTok}[1]{\textcolor[rgb]{0.00,0.00,0.81}{#1}}
\newcommand{\DocumentationTok}[1]{\textcolor[rgb]{0.56,0.35,0.01}{\textbf{\textit{#1}}}}
\newcommand{\ErrorTok}[1]{\textcolor[rgb]{0.64,0.00,0.00}{\textbf{#1}}}
\newcommand{\ExtensionTok}[1]{#1}
\newcommand{\FloatTok}[1]{\textcolor[rgb]{0.00,0.00,0.81}{#1}}
\newcommand{\FunctionTok}[1]{\textcolor[rgb]{0.00,0.00,0.00}{#1}}
\newcommand{\ImportTok}[1]{#1}
\newcommand{\InformationTok}[1]{\textcolor[rgb]{0.56,0.35,0.01}{\textbf{\textit{#1}}}}
\newcommand{\KeywordTok}[1]{\textcolor[rgb]{0.13,0.29,0.53}{\textbf{#1}}}
\newcommand{\NormalTok}[1]{#1}
\newcommand{\OperatorTok}[1]{\textcolor[rgb]{0.81,0.36,0.00}{\textbf{#1}}}
\newcommand{\OtherTok}[1]{\textcolor[rgb]{0.56,0.35,0.01}{#1}}
\newcommand{\PreprocessorTok}[1]{\textcolor[rgb]{0.56,0.35,0.01}{\textit{#1}}}
\newcommand{\RegionMarkerTok}[1]{#1}
\newcommand{\SpecialCharTok}[1]{\textcolor[rgb]{0.00,0.00,0.00}{#1}}
\newcommand{\SpecialStringTok}[1]{\textcolor[rgb]{0.31,0.60,0.02}{#1}}
\newcommand{\StringTok}[1]{\textcolor[rgb]{0.31,0.60,0.02}{#1}}
\newcommand{\VariableTok}[1]{\textcolor[rgb]{0.00,0.00,0.00}{#1}}
\newcommand{\VerbatimStringTok}[1]{\textcolor[rgb]{0.31,0.60,0.02}{#1}}
\newcommand{\WarningTok}[1]{\textcolor[rgb]{0.56,0.35,0.01}{\textbf{\textit{#1}}}}
\usepackage{longtable,booktabs}
\usepackage{calc} % for calculating minipage widths
% Correct order of tables after \paragraph or \subparagraph
\usepackage{etoolbox}
\makeatletter
\patchcmd\longtable{\par}{\if@noskipsec\mbox{}\fi\par}{}{}
\makeatother
% Allow footnotes in longtable head/foot
\IfFileExists{footnotehyper.sty}{\usepackage{footnotehyper}}{\usepackage{footnote}}
\makesavenoteenv{longtable}
\usepackage{graphicx}
\makeatletter
\def\maxwidth{\ifdim\Gin@nat@width>\linewidth\linewidth\else\Gin@nat@width\fi}
\def\maxheight{\ifdim\Gin@nat@height>\textheight\textheight\else\Gin@nat@height\fi}
\makeatother
% Scale images if necessary, so that they will not overflow the page
% margins by default, and it is still possible to overwrite the defaults
% using explicit options in \includegraphics[width, height, ...]{}
\setkeys{Gin}{width=\maxwidth,height=\maxheight,keepaspectratio}
% Set default figure placement to htbp
\makeatletter
\def\fps@figure{htbp}
\makeatother
\setlength{\emergencystretch}{3em} % prevent overfull lines
\providecommand{\tightlist}{%
  \setlength{\itemsep}{0pt}\setlength{\parskip}{0pt}}
\setcounter{secnumdepth}{5}
\usepackage{booktabs}
\ifluatex
  \usepackage{selnolig}  % disable illegal ligatures
\fi
\usepackage[]{natbib}
\bibliographystyle{apalike}

\title{Communication Theories}
\author{Mike Nguyen}
\date{2021-02-23}

\begin{document}
\maketitle

{
\setcounter{tocdepth}{1}
\tableofcontents
}
\hypertarget{prerequisites}{%
\chapter{Prerequisites}\label{prerequisites}}

This book is based on the two communications seminars

\begin{longtable}[]{@{}cc@{}}
\toprule
Course & Professor\tabularnewline
\midrule
\endhead
Interpersonal Communication & Haley Horstman\tabularnewline
Organizational Communication & Debbie Dougherty\tabularnewline
\bottomrule
\end{longtable}

Communication is defined as the exchange of messages.

\begin{Shaded}
\begin{Highlighting}[]
\FunctionTok{install.packages}\NormalTok{(}\StringTok{"bookdown"}\NormalTok{)}
\CommentTok{\# or the development version}
\CommentTok{\# devtools::install\_github("rstudio/bookdown")}
\end{Highlighting}
\end{Shaded}

\hypertarget{part-interpersonal}{%
\part{INTERPERSONAL}\label{part-interpersonal}}

\hypertarget{intro}{%
\chapter{Introduction}\label{intro}}

History

\begin{itemize}
\tightlist
\item
  Cornell School: study of speech from a humanities perspective\\
\item
  Midwestern School: study speech as a science
\end{itemize}

According to \citep{Baxter_2008}, interpersonal communication is ``the production and processing of verbal and nonverbal messages between two or a few persons''.

Three perspectives to study interpersonal communication \citep{Baxter_2008}

\begin{itemize}
\tightlist
\item
  \protect\hyperlink{individually-centered}{Individually Centered}\\
\item
  \protect\hyperlink{interactiondiscourse-centered}{Interaction/discourse Centered}\\
\item
  \protect\hyperlink{relationship-centered}{Relationship Centered}
\end{itemize}

Theory and data should be an interactive process. We should understand the conceptual boundaries of a theory, we should not apply it everywhere, generously improve it or dismiss it. Usually, there aren't one ultimate theory that has its own sovereignty \citep{Higgins_2004}. Each theory has its own assumptions. ``Making different predictions is not hte same as making competing predictions''. \citep{Higgins_2004}. a phenomenon can be explained by multiple theories, with different reasons, which shows its robustness.

A theory must be:

\begin{enumerate}
\def\labelenumi{\arabic{enumi}.}
\tightlist
\item
  Testable\\
\item
  Coherent\\
\item
  Economical/Parsimonious\\
\item
  Generalizable\\
\item
  Explanability
\end{enumerate}

A theory is like a child. Developing a theory is like parenting.

\begin{itemize}
\tightlist
\item
  Don't abuse\\
\item
  DOn't spoil\\
\item
  Knowing your theory and its limitation.
\end{itemize}

\citep[pp.~5-30]{miller1995} Assumption of interpersonal communication: " when people communicate, they make predictions about the effects, or outcomes, of their communication behavior". Prediction can be made consciously or unconsciously; hence, communication has creative element. Two sets of factors influence prediction:

\begin{itemize}
\tightlist
\item
  \textbf{situational} set: ``the given, unalterable features of a communication setting''.\\
\item
  \textbf{dispositional} set: ``our past experience and our future expectations dispose us to look for certain behaviors and to interpret them in certain ways''.
\end{itemize}

Levels of analysis used in making prediction

\begin{enumerate}
\def\labelenumi{\arabic{enumi}.}
\item
  Cultural:\\

  \begin{itemize}
  \tightlist
  \item
    culture is ``the sum of characteristics, beliefs, habits, practices, and language shared by a large group of people,''. + can be either heterogeneous or homogeneous (homogeneity increases prediction accuracy).\\
  \item
    norm is ``a recurrent, observable pattern'', which help predict behavior\\
  \item
    ideology also helps predict responses to certain messages.\\
  \item
    prediction based on cultural data can be erroneous. The more culturally diverse a society is, the more error that you will make.\\
  \end{itemize}
\item
  Sociological:\\

  \begin{itemize}
  \tightlist
  \item
    A membership group is ``a class of people who share certain common, characteristics, either by their own volition or because of some criteria imposed by the predictor''.\\
  \end{itemize}
\item
  Psychological\\

  \begin{itemize}
  \tightlist
  \item
    Sources of behavioral differences: - learning experiences\\
  \item
    reactions to experiences\\
  \item
    perception by observers of behavior.
  \end{itemize}
\end{enumerate}

``Generally know a little about a great number of people and a lot about very few people''

\[
Generalization  \\
Cultural \\
\downarrow \\
Sociological \\
\downarrow \\
Psychological 
\]

``When predictions about communication outcomes are based primarily on a cultural or sociological level of analysis, the communicators are engaged in noninterperosnla communication; when predictions are based primarily on a psychological level of analysis, the communicators are engaged in interpersonal communication''.

Cultural and sociological = noninterpersonal communication\\
Psychological = interpersonal communication.

Stimulus generalization (may have more predictive errors) vs.~stimulus discrimination.

\begin{itemize}
\item
  We make stimulus generalization initially because it is not feasible to base our prediction on psychological data.\\
\item
  very little interpersonal communication in our society:\\

  \begin{itemize}
  \tightlist
  \item
    teleological view: we should strive for interpersonal level\\
  \item
    pragmatic view: we don't need to get to the interpersonal level\\
  \end{itemize}
\item
  not every communicate interpersonally in similar ways.\\
\item
  the difference between interpersonal communication and interpersonal relationships is that in interpersonal relationship, two people must be communicating interpersonally
\end{itemize}

\citep{wilmot1995}

There are two growth trajectories for love relationships:

\begin{itemize}
\tightlist
\item
  whirlwind\\
\item
  friendship
\end{itemize}

The interpenetration of communication and relationships

\begin{itemize}
\item
  Principle 1: Relational Definition emerge from recurring episodic enactments.\\

  \begin{itemize}
  \tightlist
  \item
    An episode is ``a nonverbal and verbal communication event''.\\
  \item
    relational translation: attach relationship meaning to the episodes.\\
  \item
    ``the more frequently a relational definition is reinforced by episodic enactments, the more potent it becomes''.
  \end{itemize}
\item
  Principle 2: Relationship Definitions ``Frame'' or Contextualize Communication Behavior\\

  \begin{itemize}
  \tightlist
  \item
    :the meaning of our communication behaviors is dependent on the relational frame where they occur".\\
  \item
    ``communication is interpreted and associated within given relational definitions''.\\
  \end{itemize}
\item
  Principle 3: Relationship types are not necessarily mutually exclusive\\
\item
  Principle 4: relationship Definition and communication episodes reciprocally frame one another
\end{itemize}

\textbf{A Theory of Embeddedness}

\begin{itemize}
\item
  Relationship Constellations\\

  \begin{itemize}
  \tightlist
  \item
    definition: ``interconnected networks that form patterns''.\\
  \item
    the constellations influences initiating relationships by:\\
  \item
    the network we are in\\
  \item
    social norms\\
  \item
    the postilion of initiator and potential partner in the network\\
  \item
    direct action, or approval/disapproval by others in the network on your choice.\\
  \item
    density of the network also influences the overall constellation.\\
  \item
    not only actual actions by the constellation members that affect you, even your anticipation of the reaction of those members also affects you. \citep{Surra_1990} + people are influenced by the support or disapproval of the network\\
  \item
    ``Romeo and Juliet effect'': disapproval of parents strengthens relationship's bonds.
  \end{itemize}
\item
  Cultural Considerations
\end{itemize}

\textbf{Self and Other in relation}

\begin{itemize}
\tightlist
\item
  Self was defined as independent and autonomous.(e.g., in psychology mostly dysfunctionality exists mainly in self )
\end{itemize}

\textbf{Paradigm 1: The Individual Self}

Self and Others are ``independent units that are connected by the relational thread.'' Or mere overlap of the two separate autonomous selves who just happen to have enough in common to create a relationship."

Relationship difficulties are identified by the degree of blame of the other.

Social exchange model (assume that we try to maximize profit in relationships ). Hence, we focus on building self (self-satisfaction), not relationship.

Postmodern thinking:

Constructedness: see ``people as forming and reforming their selves within each relationship''.

\textbf{relational self}

\textbf{Paradigm 2: The Embedded Self}

``The identity of''I" is possible solely through the identity of the other who recognizes me, and who in turn is dependent upon my recognition". (Wilber, 1932, p.272)

\textbf{The Dialectical Perspective}\\
There is a dynamic interplay between opposites that we need to look at. Everything is interdependent. trade off between exactitude of factual language and seeing things in a totality way.\\
External (e.g., contradiction between autonomy and integration, me vs.~we, independence vs.~interdependence, or expressiveness vs.~protectiveness) and internal dialectical tensions in relationships

\textbf{Paradigm 3: Nonseparable self/other/relationship}

the self is the result of interaction with others.

Communication is ``a conjoint reality created by two people in relation to each other''

Paradigm I \textbar{} communication is a static, linear, noninteractive event.

Transformation = Expression + Connection

\citep{Baxter_2004}

ground relational dialectics theory:

\begin{enumerate}
\def\labelenumi{\arabic{enumi}.}
\item
  Dialogue as constitutive process\\

  \begin{itemize}
  \tightlist
  \item
    ``Communication as a conduit through which a variety of antecedent psychological ans sociological factors are played out''.\\
  \item
    Alternative: ``Communication as constitutive'': communication constitutes persons and relationships.\\
  \item
    ``An individual knows self only from the outside, as he or she conceives others see him or her. The self, then, is invisible to itself and dependent for its existence on the other''. Hence, self is ``a fluid and dynamic relation between self and other''. + self-becoming resembles self-expansion model.
  \end{itemize}
\item
  Dialogue as dialectical flux\\

  \begin{itemize}
  \tightlist
  \item
    Dialogue is ``simultaneously unity and difference''. hence, social life is a dialogue ``constituted in the dialectical, or contradictory, interplay of centripetal and centrifugal forces''.\\
  \item
    contrast to Hegelian approach to dialogue\\
  \end{itemize}
\item
  Dialogue as aesthetic moment\\
\item
  Dialogue as utterance\\
\item
  Dialogue as critical sensibility
\end{enumerate}

Braithwaite's Perspectives on interpersonal communication

\begin{itemize}
\tightlist
\item
  Numerical Perspective\\
\item
  Situational and contextual perspective\\
\item
  Developmental Perspective\\
\item
  Levels of Info Perspective \citep{miller1995}\\
\item
  Relational (Stewart) focusing on the content.
\item
  Constitutive Approach \citep{Baxter_2004}
\end{itemize}

Def of IPC = when predictions about comm outcomes are based primarily on a psych level of analysis (p.~22)\\
* IPC occurs when:\\
1. Predictions are based on personal level info\\
2. Have direct experience with other person\\
3. Initial interactions are rarely interpersonal\\
4. Most interactions are non-interpersonal\\
5. Relationships exist when both people are communicating interpersonally

Chapter 2 (Stewart (2019))

Communication is ``the processes humans use to construct meaning together''.

\begin{enumerate}
\def\labelenumi{\arabic{enumi}.}
\tightlist
\item
  Since humans live in worlds of meaning that are constantly constructed, none can affect the process significantly.\\
\item
  Culture figures (ethnicity, gender, age, social class, sexual orientation, etc) affect communication and how you respond to it.\\
\item
  we collaboratively build the sense of selves (i.e., identity) when engaging in communication.\\
\item
  Conversations are a tools for communication.\\
\item
  A useful skills in communicating is ``nexting''.
\end{enumerate}

Communication is ``the continuous, complex, collaborative, process of verbal and nonverbal meaning-making through which we construct the worlds of meaning we inhabit.''

Worlds of meanings:

\begin{itemize}
\tightlist
\item
  space
\item
  time
\item
  laws of physics\\
\item
  culture
\item
  relationships
\item
  work (for adults).
\end{itemize}

Interpersonal Communication:\\
``people involved are contacting each other as persons''

Characteristics that distinguish persons across cultures:

\begin{itemize}
\tightlist
\item
  uniqueness: noninterchangeability (either experiential or genetic)
\item
  unmeasurability: human can't be described by parts. even though cognitive scientists try to assign schematas or cognitive patters/ Emotions and feeling are embedded in communications.\\
\item
  Responsiveness is different from reaction.
\item
  reflectiveness: being aware of what's around, but also aware of your own awareness.
\item
  addressability: difference between talking to and talking with (i.e., addressable). directed or aimed at.
\end{itemize}

``the term interpersonal labels the kind of communication that happens when the people involved talk and listen in ways that maximize the presence of the personal''.

\citep{Floyd_2014}

Interpersonal communication is defined as ``Any communication at the intrapersonal, small group, public, or mass levels.''

Boundary condition includes:

\begin{itemize}
\tightlist
\item
  dyad relationships\\
\item
  ``IPC as close, supportive, relationship-maintaining communication occurring between people (whether in a dyad or not''
\end{itemize}

\hypertarget{individually-centered}{%
\chapter{Individually Centered}\label{individually-centered}}

\hypertarget{uncertainty-management-theories}{%
\section{Uncertainty Management Theories}\label{uncertainty-management-theories}}

\hypertarget{problematic-integration-theory}{%
\subsection{Problematic Integration Theory}\label{problematic-integration-theory}}

Problematic Integration (PI) theory: From the theories of planned behavior and reasoned action, we believe that we can predict people's behaviors because people are assumed to be ``rational''. However, there are communication substance that could input uncertainty and inconsistency expectations to predict human behavior.

\begin{itemize}
\item
  Goals:

  \begin{itemize}
  \tightlist
  \item
    find important and ubiquitous communication process\\
  \item
    increase sophistication\\
  \item
    encourage other ways of understanding\\
  \item
    increase communicators' empathy and compassion.\\
  \end{itemize}
\item
  Forms of PI:

  \begin{itemize}
  \tightlist
  \item
    Uncertainty\\
  \item
    Diverging expectations and desires\\
  \item
    Ambivalence\\
  \item
    Impossible desires (theoretical vs.~practical impossibility).\\
  \end{itemize}
\item
  Discussion regarding PI can deepen or hurt relationships\\
\item
  Encounter PI, we can engage in presentational and avoidance rituals.\\
\item
  PI defines uncertainty as ``difficulty forming a mental association''. \citep{Babrow_2009}

  \begin{itemize}
  \tightlist
  \item
    form-specific adaptation of messages means ``communicating in ways that speak to the precise dilemma.'' \citep{Babrow_2009}
  \end{itemize}
\end{itemize}

\hypertarget{uncertainty-management-theory}{%
\subsection{Uncertainty Management Theory}\label{uncertainty-management-theory}}

Uncertainty Management (UM)

\begin{itemize}
\item
  Based on two post-positivist sources:

  \begin{itemize}
  \tightlist
  \item
    Uncertainty reduction theory \citep{BERGER_1975}: managing uncertainty\\
  \item
    Cognitive theory of uncertainty in illness \citep{Mishel_1990}: depending on context, uncertainty can be either good or bad\\
  \end{itemize}
\item
  Uncertainty must be appraised.\\
\item
  Notion of management = control
\end{itemize}

Research and practical application (e.g., health, education, )\\
Evaluation: not achievable under post-positivist because of its blurry boundary conditions. But under interpretivist, it can make more sense due to its contextual meanings.

Application:\\
Taking Control: The Efficacy and Durability of a Peer-Led Uncertainty Management Intervention for People Recently Diagnosed With HIV \citep{Brashers_2016}: Uncertainty management need to be adaptable. Due to the changing nature of HIV skills and information for patients need to be communicated continuously. Supported by the theories of social support, uncertainty management can be facilitated with peer support. participant report less illness-related uncertainty, greater access to social support, and more satisfaction with the social support compared to the control group. \textbf{Illness uncertainty} was assessed with \citep{MISHEL_1981}.

Example

\citep{SHARABI_2017} Effects of the first FtF date on romantic relationship development:

\begin{itemize}
\tightlist
\item
  Relational choice models of romantic relationships: Choosing partners that make the most sense to you (fit an image of an ideal mates).\\
\item
  Disillusionment models of romantic relationship: When you see other's aspects (e.g., personality, behaviors) of your partner, you might no longer be interested in your partner.
\end{itemize}

Predicting first date success in online dating

\begin{itemize}
\tightlist
\item
  Similarity and uncertainty as predictors: users want to reduce uncertainty before meeting offline.\\
\item
  Communication as moderating role.
\end{itemize}

Interestingly, people disclose more deeply online compared to offline \citep{Tidwell_2002}

\hypertarget{theory-of-motivated-informaiton-management-tmim}{%
\subsection{Theory of Motivated Informaiton Management (TMIM)}\label{theory-of-motivated-informaiton-management-tmim}}

Born from the frustration with \protect\hyperlink{problematic-integration-theory}{Problematic Integration Theory}, \protect\hyperlink{uncertainty-management-theory}{Uncertainty Management Theory} interepretivist orientation, and desire to incorporate individual experience's complexity with uncertainty and predictive specificity.

The theory has its basis on:

\begin{itemize}
\tightlist
\item
  Subjective Expected Utility theory \citep{Fischhoff_1983}\\
\item
  social Cognitive theory \citep{Locke_1987}
\item
  Theories of bounded rationality \citep{Kahneman_2003}: People make suboptimal choice due to other emotions and bias factors.
\end{itemize}

Due to its laborious process of decision, theory of motivated information management only applies to cases where the person thinks a decision is sufficient important.

Phases:

\begin{itemize}
\item
  Interpretation Phase: recognize the difference (called \textbf{uncertainty discrepancy}) in desired uncertainty and current uncertainty, which mostly produces anxiety, but sometimes hope, anticipation, anger.\\
\item
  Evaluation Phase: :appraisal of uncertainty impacts assessments made in the evaluation phase", which makes you think about

  \begin{itemize}
  \item
    Outcome expectancy: what happen if you search for more info\\
  \item
    Efficacy: whether you are able to do the search.

    \begin{itemize}
    \tightlist
    \item
      Communication efficacy: whether a person has the skill to seek info.\\
    \item
      Target efficacy: whether the target of the info search actually has and would be willing to share it.\\
    \item
      Coping efficiency: whether a person could emotionally, relational, or financially deal with what he or she expects to learn.
    \end{itemize}
  \end{itemize}
\item
  Decision Phase: people are likely to seek info when they expect positive outcomes with high levels of efficacy.
\end{itemize}

\begin{center}\includegraphics[width=11.11in]{images/Model of TMIM Predicitons} \end{center}

(picture from \citep{Baxter_2008})

Note: Information providers go through the same process with only the latter two phases (evaluation and decision).

Research and Practical Application: (e.g., education, health)

Evaluation:

\begin{itemize}
\item
  Benefits:

  \begin{itemize}
  \tightlist
  \item
    Draw attention to communication efficiency, and outcome expectancy\\
  \item
    Good theory: based on testability, heuristics, parsimony, scope condition\\
  \end{itemize}
\item
  Improvement:

  \begin{itemize}
  \tightlist
  \item
    may need to include efficacy's strength as mediator. Depending on the positivity or negativity of expectations. relationship between outcome expectancies and efficacy, and between outcome expectancies and information seeking may differ
  \end{itemize}
\end{itemize}

Example:

\citep{Morse_2013} social networks and information seeking influence drug use. From Social Cognitive Theory, and Cognitive Developmental Theory, social norms and peer influence serve as bases for aversive behaviors to be accepted. According to \citep{Wolfson_2000}, false consensus support can help explain students overestimate of the positive attitudes of their social network supported by the fact that they are uncertain about their social network's opinions.

\hypertarget{attribution-theory}{%
\section{Attribution Theory}\label{attribution-theory}}

``how and why we try to answer''how and why" questions is referred to as attribution theory" \citep{Baxter_2008}

originated from psychology. ``The more important or unexpected the event, the more likely people are to seek an explanation to make sense of that outcome. We make sense of such events primarily by determining what the cause is.''

Goals

\begin{itemize}
\tightlist
\item
  Event causation: understand actions or events by attributing cause(s) to behavior.
\item
  Trait inference: make inference about a person' characteristics that makes sense of that person's behavior.
\end{itemize}

Dimensions when making attributions:

\begin{itemize}
\tightlist
\item
  locus: interval or external to the person\\
\item
  Stability: temporary or enduring\\
\item
  Specificity: causes is unique or universal\\
\item
  Responsibility: the extent to which a person contribute to the event
\end{itemize}

Focus on:

\begin{itemize}
\item
  Correspondence: ``When attributions are informative of a person's nature or personality, they are considered \textbf{"correspondent"} (i.e., we perceive that another's behavior corresponds to some underlying characteristic of who that person is)''.\\
\item
  Covariation: ``Events are attributed to causes with which they covary.''
\item
  Responsibility: the more internal, intentional, and controllable we perceive one's behavior is, the more we hold that person responsible for those actions, and their consequences"\\
\item
  Bias:

  \begin{itemize}
  \tightlist
  \item
    ``fundamental attribution bias, which is a tendency to make more internal attributions than external attributions for other people's behaviors'' \citep{Ross_1977}\\
  \item
    self-serving bias: people generally make more internal, stable, and global attributions for positive events than for negative events, and more external attributions for negative events than for positive events \citep{Malle_2006}
  \end{itemize}
\end{itemize}

Attribution Theory in Communication:

\begin{itemize}
\tightlist
\item
  Attribution as Explanations behind social communicative actions.\\
\item
  Attribution as reason for actions and outcomes: when we think of reasons for other's communication or behaviors, it affects how we view others, and our communication toward them.\\
\item
  Attribution as the meanings given to a behavior: ``how attributions reflect the meaning that people give to a communication act.''
\end{itemize}

Evaluation:

\begin{itemize}
\tightlist
\item
  Explanatory power: intuitive\\
\item
  Scope and generality: applicability, born as universal theory of human sense-making, but actual application was limited\\
\item
  Conditionship specification: strict parameters for the theory.\\
\item
  Verifiability/ Falsifiability: a lot of research supports, few say the theory is flawed.
\end{itemize}

\hypertarget{social-exchange-theories}{%
\section{Social Exchange Theories}\label{social-exchange-theories}}

Costs vs.~Rewards.

Originated from psychology, sociology, economics. Analogous to economic exchange. Under the post-positivist paradigm.

Definitions:

\begin{itemize}
\tightlist
\item
  An exchange is ``a transfer of something in return for something else'' \citep{Leffler_1982}\\
\item
  Social exchange is the result of human's connection.
\end{itemize}

\begin{longtable}[]{@{}lll@{}}
\toprule
Aspect & Social Exchange & Economic Exchange\tabularnewline
\midrule
\endhead
Reliance & Trust, goodwill, voluntary & Legal Obligations\tabularnewline
Rewards and Costs & Open & Exact Specifications for both parties\tabularnewline
Time frame & Continuous & Set, fixed for the exchange to occur\tabularnewline
Type & Unique, individualized & Similar from person to person\tabularnewline
\bottomrule
\end{longtable}

Goals:

\begin{itemize}
\tightlist
\item
  Predict and explain behaviors.
\end{itemize}

Assumptions:

\begin{itemize}
\tightlist
\item
  Social behavior is a series of transactions.\\
\item
  ``Individuals attempt to maximize their rewards and minimize their costs.''\\
\item
  After receiving rewards, people feel a sense of obligation.
\end{itemize}

Concepts:

\begin{itemize}
\tightlist
\item
  Self-interests: ``individuals to act in accordance with perceptions and projections of rewards and costs associated with an exchange, or potential exchange, of resources.'' we are motivated to serve self-interests.\\
\item
  Interdependence: ``the extent to which one person's outcomes depend on another person's outcomes''
\end{itemize}

Social Exchange in Communication:

\begin{itemize}
\tightlist
\item
  communication is a communication tool
\item
  communication is the resource to be exchange (i.e., either reward or cost).\\
\item
  Exchange may have symbolic or communication value \citep{Molm_2007}
\end{itemize}

Evaluation:

\begin{itemize}
\tightlist
\item
  love can be selfless: \textbf{Altruism} is beyond social exchange\\
\item
  High in exchange orientation are likely to keep score \citep{Murstein_1971}\\
\item
  Cultures differ in their exchange orientations: exchange orientation is more expected in individualistic and capitalistic societies. \citep{Van_Yperen_1990}\\
\item
  People are not also rational (scale of inequity is not always instantly balanced)
\end{itemize}

Application:

\begin{itemize}
\tightlist
\item
  emotional health (individual), trusting one's spouse (interpersonal), and \textbf{feeling underbenefited in the relationship (interpersonal)} significantly predict marital well-being for both groups of women (i.e., African American and European American). While physical health (individual) and in-law relations (social and economic) showed significant influence for only African American \citep{Goodwin_2003}.
\end{itemize}

\hypertarget{resource-theory}{%
\subsection{Resource Theory}\label{resource-theory}}

``Resources constitute rewards when they provide pleasure and costs when they provoke pain, anxiety, embarrassment, or mental and physical effort.''

Developed by \citep{Foa_1980, Foa_2012}

Types of resources:

\begin{itemize}
\tightlist
\item
  Money: universal\\
\item
  Goods\\
\item
  Status\\
\item
  Love\\
\item
  Services\\
\item
  Information
\end{itemize}

Exchange of similar resources results in more satisfaction \citep{Foa_1980}. And relationship type influences the exchange of resources.

\hypertarget{interdependence-theory}{%
\subsection{Interdependence Theory}\label{interdependence-theory}}

Individuals assess their rewards in a relationship based on

\begin{itemize}
\tightlist
\item
  Comparison levels: what one \emph{should} receive: ``the standard an individual uses to judge how attractive or satisfactory a particular relationship is.'' Relate to \textbf{normative economics}
\item
  Alternatives (Comparison levels of alternatives): what one \emph{could} receive: ``the lowest level of rewards deemed acceptable when considering possible alternative relationship.''
\end{itemize}

Note:

\begin{itemize}
\tightlist
\item
  Our projection is not always right. For example, the more committed and invested we are in a relationship, the more likely we are to downplay alternatives \citep{Rusbult_2010}
\end{itemize}

Application:

\begin{itemize}
\tightlist
\item
  \citep{Vangelisti_2013}: correlation between individuals' cognition and their relational satisfaction. Individuals' vocalized thoughts correlate with their partner's satisfaction.
\item
  equity and satisfaction (under the interdependence theory ) influences one's relational maintenance strategies \citep{Stafford_2006}
\end{itemize}

\hypertarget{equity-theory}{%
\subsection{Equity Theory}\label{equity-theory}}

We also consider \textbf{fairness} in our equation of gains and costs, where fairness is ``equity in the distribution of costs and rewards''\citep{Baxter_2008}.

\textbf{Distributive justice} \citep{Adams_1965}: ``people think and act so that rewards are distributed in accordance with their effort.'' Three types of inequity:

\begin{itemize}
\tightlist
\item
  ratio of your rewards to costs in vs.~others' ratios.\\
\item
  ``the exchange relationship you and your partner have with a third entity''\\
\item
  your relationship vs others in similar situation.
\end{itemize}

Inequity leads to emotional distress \citep{Sprecher_2001}. Underbenefitied experiences anger, whereas overbenefited experiences guilt. To balance our inequity, we change outcomes (perceptions), or inputs (actions)

Application:

\begin{itemize}
\tightlist
\item
  Perceptions of equity influences caregiver burnout, and positive caregiver experiences \citep{Ybema_2002}
\end{itemize}

\hypertarget{social-support-theories}{%
\section{Social Support Theories}\label{social-support-theories}}

Supportive communication is ``verbal and nonverbal behavior produced with the intention of providing assistance to others perceived as needing that aid.'' \citep[pp.317]{MacGeorge_2011}

Verbal person centeredness (VPC), defined as "the extent to which the feelings and perspective of a distressed other are acknowledged, elaborated, and legitimized: \citep{MacGeorge_2018}. However, research sometimes use VPC for the entire interaction, or advisors or recipients.

Person centeredness is defined as ``awareness of and adaptation to the subjective, affective, and relational aspects of communicative contexts'' \citep[pp.~249]{burleson_1998}.

Dimensions of support behavior:

\begin{itemize}
\item
  content (i.e., topical focus)
\item
  function (i.e., observed (inferred) intention of the provider/advisor) (e.g., describing, legitimizing, minimizing, recommending, justifying, blaming, criticizing, questioning, affirming, encouraging, and offering tangible support)
\item
  experiential focus (i.e., ``the person whose experiences are being referenced in the supportive behavior'' \citep[pp.~153]{MacGeorge_2018}
\end{itemize}

\hypertarget{dual-process-theory-of-supportive-message-outcomes}{%
\subsection{Dual-Process Theory of Supportive Message Outcomes}\label{dual-process-theory-of-supportive-message-outcomes}}

Comes from the dual-process model in psychology: ``People actions are a function of the ways in which they interpret or make sense of events.'' \citep[pp.106]{burleson_2010}

Goals and Features:

\begin{itemize}
\item
  ``the impact of messages varies as a function of how those messages are processed, and it provides a detailed analysis of the processing modes that can be applied to supportive messages.''\href{https://www.semanticscholar.org/paper/Understanding-the-outcomes-of-supportive-A-approach-Burleson/34f073a64a9d4e5e092d816202ee415768ceb26e}{\includegraphics{images/9-Figure1-1.png}}

  \citep[pp.198]{Baxter_2008}
\end{itemize}

Modes:

\begin{itemize}
\item
  Processing modes: Elaboration (i.e., ``the extent to which an individual thinks with respect to message content'')

  \begin{itemize}
  \item
    negative affect
  \item
    motivation
  \item
    ability
  \item
    environmental cues
  \item
    Quality of supportive message: high vs.~low
  \end{itemize}
\end{itemize}

Under the framework of dual-process theory, communication is defined as ``a process in which a person (the source) seeks to convey or make public some internal state to another (the recipient) through the use of signals and symbols (the message) in the effort to accomplish some pragmatic end (the goal).'' \citep{burleson_2010}

Application:

\begin{itemize}
\item
  emotional support
\item
  grief management
\end{itemize}

\citep{Davis_2018} studies the microaggression of white women towards black women with two phases: Individual orientation phase (i.e., ``friends communicating verbal and nonverbal messages that solely comforted the support seeker'' - information seeking, support provision (e.g., the use of girls, hand clap)), and Collective orientation phase (phase: Hostile differentiation, Socio-political Contextualization, Collective Uplift). Age moderates the perceived microaggression (e.g., tolerance).

Racial microaggressions are ``brief messages (i.e., verbal, nonverbal, and visual) that denigrate people of color because they belong to a racial group that is historically oppressed in the U.S.'' \citep{Sue_2007}

Strong Black Woman Collective Theory argues that ``strength is valuable resource for Black women because it helps them resist external hostilities.'' \citep{Davis_2014}

\hypertarget{advice-response-theory}{%
\subsection{Advice Response Theory}\label{advice-response-theory}}

Social cognitive theory: how advice outcomes are influenced by qualities of messages, advisors, situations, and recipients.

\textbf{Goals}:

ART predicts how your friend is likely to respond, based on your friend's perceptions of

\begin{enumerate}
\def\labelenumi{\arabic{enumi}.}
\item
  Message features (e.g., content and style): Recipients evaluate

  \begin{enumerate}
  \def\labelenumii{\arabic{enumii}.}
  \item
    message content

    \begin{itemize}
    \tightlist
    \item
      efficacy (i.e., if the action is likely to resolve the problem)
    \item
      feasibility (i.e., capacity to accomplish eh action)
    \item
      limitation
    \item
      confirmation (whether he action is consistent with the recipient's intent)
    \end{itemize}
  \item
    Style:

    \begin{itemize}
    \tightlist
    \item
      politeness
    \item
      linking
    \item
      respect
    \end{itemize}
  \end{enumerate}
\item
  Advisor's characteristics (likely to be mediated by message content)

  \begin{itemize}
  \tightlist
  \item
    Expertise (to the problem)
  \item
    trustworthiness
  \item
    likability
  \item
    similarity (to the recipient).
  \end{itemize}
\item
  Situational factors (this is controversial because of conflicting empirical evidence)

  \begin{itemize}
  \tightlist
  \item
    problem seriousness (perceived by the recipient)
  \item
    solution uncertainty (about how to resolve the problem)
  \end{itemize}
\item
  Recipient's traits or characteristic

  \begin{itemize}
  \tightlist
  \item
    thinking style
  \item
    abilities (e..g, cognitive complexity)
  \item
    demographic (e.g., culture, gender)
  \end{itemize}
\end{enumerate}

Application

\begin{itemize}
\tightlist
\item
\end{itemize}

\hypertarget{interactiondiscourse-centered}{%
\chapter{Interaction/discourse Centered}\label{interactiondiscourse-centered}}

\hypertarget{evolutionary-theories}{%
\section{Evolutionary Theories}\label{evolutionary-theories}}

\hypertarget{intergroup-theorizing}{%
\section{Intergroup Theorizing}\label{intergroup-theorizing}}

\hypertarget{critical-approcahes-to-ipc-resarch}{%
\section{Critical Approcahes to IPC Resarch}\label{critical-approcahes-to-ipc-resarch}}

\hypertarget{relationship-centered}{%
\chapter{Relationship Centered}\label{relationship-centered}}

\hypertarget{affection-exchange-theory}{%
\section{Affection Exchange Theory}\label{affection-exchange-theory}}

\hypertarget{traslational-scholarship}{%
\section{Traslational Scholarship}\label{traslational-scholarship}}

\hypertarget{resilience-communciation-theories}{%
\section{Resilience Communciation Theories}\label{resilience-communciation-theories}}

\hypertarget{communciation-privacy-management-theory}{%
\section{Communciation Privacy Management Theory}\label{communciation-privacy-management-theory}}

\hypertarget{relational-turbulence-theory}{%
\section{Relational Turbulence Theory}\label{relational-turbulence-theory}}

\hypertarget{part-organizational-communication}{%
\part{ORGANIZATIONAL COMMUNICATION}\label{part-organizational-communication}}

\hypertarget{perspectives-on-organizational-communication}{%
\chapter{Perspectives on Organizational Communication}\label{perspectives-on-organizational-communication}}

\citep{Mumby_1996}
Organizational communication as a discipline can be looked under the framework of 4 problematics.

The problematic of:

\begin{enumerate}
\def\labelenumi{\arabic{enumi}.}
\tightlist
\item
  voice: characterized by multiple voices, not only managerial.\\
  + organizational communication cultivates tensions between university and firms, rather than resolving it.\\
  + how voices can gain insight into marginalized groups.\\
\item
  rationality\\
  + pluralist understandings
  + technical rationality: ``knowledge that privileges a concern with prediction, control and teleological forms of behavior''.\\
  + Practical rationality: ``knowledge grounded in the human interest in interpreting and experiencing the word as meaningful and intersubjectively constructed''\\
\item
  organization\\
  + The question of organization is fundamental in organizational communication.\\
  + the complex structure of organizing, culture and larger social processes.\\
\item
  organization-society relationship\\
  + organizational boundaries (separation between organization and society) cannot be clearly defined due to its fluid nature.\\
  + can study the dynamics nature of globalization.\\
  + communication is not just information exchange, but it is the core of organizing where organization structure is dynamically created.
\end{enumerate}

\citep{Broadfoot_2007}~\\
We might have been myopic when only interpret and look at organizational communication from the perspective of Euro-American intellectual tradition. hence, we need to have alternative, rationalities, and perspectives.

Due to \citep{Mumby_1996}, there are four major problematics in organizational communication:

\begin{itemize}
\tightlist
\item
  Voice: who gets to speak for whom\\
\item
  Rationality: 2 forms of rationalities: technical/instrumental and practical and the consequences.\\
\item
  Organization: members create meaning through communication.\\
\item
  Organization-society relationship: it's hard to distinguish between the two, hence we should study in conjunction.
\end{itemize}

there is a new shift to the non-American voices: A Postcolonial awakening.

Postcolonial self-reflexivity: a resistance from Eurocentric perspective.

\citep{Shome_1996} defines Discursive confinement as ``a state where difference and individuality are eased or neutralized and scholars become confined to a narrow and marginalized discursive space constructed by dominant mainstream structures and ideologies''. Hence, we should break through the discipline and embed individuality through emotionality.

We can see the shift in areas such as gender, race, and globalization.

A postcolonial exploration: different perspective can contribute richly to the understanding organizational communication.

\citep{Cheney_2007}

Identity: from business. flow of information between stakeholders.

Breaking boundaries: expand to other issues such as informal network, social movements, etc.

Opportunities from social problems: shift from basic research to focus on society and planet.

Ethos and Confidence: The discipline of organizational communication as well as communication are constantly in need to prove for its legitimacy.

Audiences: various outlets, but mostly focus on research publication due to the need for tenure.

To get beyond the pressure for tenure, the author suggests:

\begin{itemize}
\tightlist
\item
  choose an issue that you care.\\
\item
  listen/read well from various perspectives\\
\item
  choose appropriate outlets.\\
\item
  set everyday goal.\\
\item
  practice what you preach\\
\item
  lead by example\\
\item
  do not give up\\
\item
  pause and reflect.
\end{itemize}

\citep{DUrso_2014}

History (genealogy) of organizational communication with the method of network analysis.

Author posted several research questions that could use the network analysis method to probe into such as collaboration and coauthorship, and overall development of organizational communication.

\citep{Leonardi_2016}

the strategy of subordination taken by organizational communication researchers are those that look at a phenomena from the perspective of organizational communication, which leads to small contribution to the literature.

To know if a one owns a phenomenon is when people know to turn to you when they wan to understand such phenomenon .

\hypertarget{strategy-of-discovery}{%
\subsection{Strategy of Discovery}\label{strategy-of-discovery}}

2 steps:

\begin{enumerate}
\def\labelenumi{\arabic{enumi}.}
\tightlist
\item
  Phenomenon is communication\\
\item
  What communication does and why
\end{enumerate}

\hypertarget{strategy-of-reconceptualizatinon}{%
\subsection{Strategy of Reconceptualizatinon}\label{strategy-of-reconceptualizatinon}}

2 steps:

\begin{enumerate}
\def\labelenumi{\arabic{enumi}.}
\tightlist
\item
  Contradictory evidence or poor explanation\\
\item
  Communications leads to better fit (e.g., accuracy or novelty)
\end{enumerate}

\hypertarget{organizational-culture}{%
\chapter{Organizational Culture}\label{organizational-culture}}

\citep{Martin_1983}

Culture:

\begin{itemize}
\tightlist
\item
  based on history, members can behave and expected to behave\\
\item
  help construct common value for employees.\\
\item
  control mechanisms which dictate patterns of behavior
\end{itemize}

culture can hardly be under control, not monolithic phenomenon

3 levels of culture:

\begin{itemize}
\tightlist
\item
  basic assumptions\\
\item
  values/ideology\\
\item
  artifacts (e.g., stories, rituals, dress): express values\\
\item
  management practices (e.g., training program).
\end{itemize}

Types of subcultures:

\begin{itemize}
\tightlist
\item
  enhancing: same position\\
\item
  orthogonal: unrelated position\\
\item
  counterculture: opposite position: ``most likely to arise in a strongly centralized institution that has permitted significant decentralization of authority to occur'' (e.g., GM's culture: team players, loyalty, ``refrigerator story''), balancing act must be taken to manage counter culture and dominant culture
\end{itemize}

\citep{Dixon_2009}
multiple meanings of organizational culture

Consulting method: in-depth and focus group interviews with student staff, artifact analysis, and observation of organization staff meetings and retreats

Common terms did not mean the same thing. 2 different fields: organizational communication, and higher education.

\begin{itemize}
\tightlist
\item
  Organization culture: ``German approach, based in phenomenological/Interpretive epistemology''. culture is the product of symbolic interaction. Scholars tries to understand the role of human interaction. organizational culture is not easily manipulated by managers. " organization is a culture". purposes:\\
  + increasing productivity\\
  + understanding organizational processes\\
  + critiquing oppressive organizational practices.
\item
  organizational culture: American approach to study organizational variable that affect organizational effectiveness. ``organization has a culture''. can be quantified, and manipulated.\\
  + Institution can be measured: dynamism vs.~stability and internal vs.~external focus.
\end{itemize}

two subculture: First-born (tradition, consensus) and Youngest (debate, and new ideas)

The problem stems from different discipline understanding of ``culture'', there was a rejection of the definition by organizational communication scholars.

" Rather than positing that there is one ``right'' concept, we would encourage other consultants to proactively discuss with clients, what key terms mean to them in the particularity of their context, as a means of creating a ``shared discursive'' reality."

\citep{Leonardi_2008}
mergers between two technology companies

cultural studies of postmerger integration

A core technology is ``the primary technology produced, serviced, or sold by an organization''.

technological grounding suggests that ``an organization's core technologies are, along with the work and communication practices enacted daily by members, a constitutive feature of its culture''

two dominant perspectives for understanding culture that exist in organizational literature:

\begin{itemize}
\tightlist
\item
  as a variable that can be changed.\\
  + technology is a variable . The two variables are distinct and can be either internal or external based on researchers' perspective.\\
\item
  culture is organization.\\
  + in postmerger, organizations face cultural convergence.
  + technology is not a variable but a practice.
  + ``When technologies are sufficiently important to an organization to become key elements in the constitution of a culture, we refer to that organization as technologically grounded.'' (a continuum not dichotomy).\\
  + ``technological incompatibility implies the incompatibility of organizational cultures and practices''
\end{itemize}

Method: a single case design, embedded design:

levels of analysis\\
(1) public discourse from company officials about the merger,
(2) organizational practices and policies before and after the merger
(3) worker responses during postmerger integration

US West built its culture on the West culture use analog data\\
Qwest built its culture on speed use digital data (all internet protocol - IP)

Qwest consumed US West's culture (e.g., bureaucracy) due to its technological superiority and cultural superiority in postmerger integration

Qwest shut down US West's Research Labs.

\hypertarget{section}{%
\section{4}\label{section}}

Chapter 4: Communicating Organizational Culture: A Problem-Solving Model.

Communication: is about creating message, production and reproduction of meaning.

Organizations are communication.

Gestalt Theory (figure and ground): sometimes the important part is thought of as the background

Organizational culture is an active process that shape organizations.

organizational culture is defined ``as the shared communicative process through which meanings are constantly employed, negotiated, and contested to create a stable communication environment within which organizational life becomes patterned and persistent over time.''

organizational cultures does not mean shared meaning but \textbf{shared process of meaning making}.

Forms of communication:

\begin{itemize}
\tightlist
\item
  info sharing\\
\item
  message production\\
\item
  meaning making
\end{itemize}

organizational values as ``those things, standards, and ideals through which we evaluate our organizational wellbeing''.

Types of values:

\begin{itemize}
\tightlist
\item
  Personal values\\
\item
  Moral values\\
\item
  Aesthetic values\\
\item
  Status values: power allocation.
\end{itemize}

Organizational meanings

\begin{itemize}
\tightlist
\item
  Cognitive meanings\\
\item
  Emotional meanings: people might mistakenly consider irrationality as emotionality.\\
\item
  Social meanings sensemaking theory\\
\item
  Identity meanings cultural contract theory of identity. 3 types of cultural contracts:\\
  + ready-to-sign contracts: assimilation (physical, behavioral,a nd mental assumption of dominant culture).\\
  + Quasi-completed contracts: allows adaption\\
  + Cocreated contracts: mutual valuation.\\
\item
  Power meanings\\
  + can derived from formal hierarchy\\
  + or from relationships (as opposed to isolation).
\end{itemize}

\hypertarget{sensemaking}{%
\chapter{Sensemaking}\label{sensemaking}}

Sensemaking is ``how organizational members come to understand and move forward when faced with unexpected or unanticipated information'' (Dougherty, 2020)

\begin{itemize}
\tightlist
\item
  It can help stabilize the organization in time of crisis.
\end{itemize}

The difference between sensemaking theory and \protect\hyperlink{uncertainty-management-theory}{Uncertainty Management Theory}: they are close ties.

\begin{longtable}[]{@{}ll@{}}
\toprule
\begin{minipage}[b]{(\columnwidth - 1\tabcolsep) * \real{0.50}}\raggedright
Uncertainty management Theory\strut
\end{minipage} & \begin{minipage}[b]{(\columnwidth - 1\tabcolsep) * \real{0.50}}\raggedright
Sensemaking Theory\strut
\end{minipage}\tabularnewline
\midrule
\endhead
\begin{minipage}[t]{(\columnwidth - 1\tabcolsep) * \real{0.50}}\raggedright
based on individual level\strut
\end{minipage} & \begin{minipage}[t]{(\columnwidth - 1\tabcolsep) * \real{0.50}}\raggedright
group dynamics and group behavior\strut
\end{minipage}\tabularnewline
\begin{minipage}[t]{(\columnwidth - 1\tabcolsep) * \real{0.50}}\raggedright
management in relationship uncertainty\strut
\end{minipage} & \begin{minipage}[t]{(\columnwidth - 1\tabcolsep) * \real{0.50}}\raggedright
manage in the organizational context\strut
\end{minipage}\tabularnewline
\bottomrule
\end{longtable}

\begin{center}\includegraphics[width=1\linewidth]{images/Sensemaking} \end{center}

(picture from \citep{Lu_2017})

Example in business: \citep{KennethWm_2014}

\citep{Dougherty_2004}

Culture of sexual harassment (i.e., Some cultures are more prone to sexual harassment than others).

From the perspective of sensemaking theory, organizational members make sense of unexpected events through a process of action, selection and interpretation \citep{Weick_1995}.

Organizational culture is created \textbf{not} through shared meaning, but \textbf{shared experiences through processes sensemaking}.
We might never come to a consensus, but the process of sensemaking can help us have shared experiences.

Properties of sensemaking:

\begin{itemize}
\tightlist
\item
  Identity: created through the interaction with other organizational members.
\item
  Retrospective: make sense only looking backward.
\item
  Ongoing: relate past, present, and future to make sense of an event.
\item
  Enactment: actors are part of the culture.
\item
  Extracted cues: focus their attention to parts of the environment.
\item
  Social: based on either interaction with others, or expected interaction with others.
\item
  Plausibility: seems reasonable.
\end{itemize}

Hence, sensemaking influence

\begin{itemize}
\tightlist
\item
  the acceptance of sexual harassment in an organization
\item
  responses by nonharassed members.
\end{itemize}

Sensemaking's phases:

\begin{itemize}
\tightlist
\item
  Discovery
\item
  Debriefing (e.g., humor, ridicule in case of sexual harassment)
\item
  Dispersal (e.g., return to normalcy)
\end{itemize}

men and women make of sexual harassment differently (i.e., women label more behavior as sexual harassment than men)

Practical Applications

\begin{itemize}
\tightlist
\item
  Applying Humor: humor can help members involve actively in sharing sexual harassment training, sense of community. But too much can also belittle victim's experience.
\item
  White men and sexual harassment: should to vilify, but assume that they want to help.
\item
  Identifying Sexual harassment: should not focus on shared meaning, but shared experience.
\item
  Responding to sexual harassment: no one-size-fit-all approach, but respect contexts of the sexual harassment.
\end{itemize}

\citep{Shenoy_Packer_2014}

First-generation immigrants are prone to microaggressions.

microaggressions are ``brief and commonplace daily verbal, behavioral, or environmental indignities, whether intentional or unintentional, that communicate hostile, derogatory, or negative racial slights and insults''.
\citep{Sue_2007}

Microagresssion exists in 3 forms:

\begin{itemize}
\tightlist
\item
  Verbal: Sarcasm
\item
  Attitudinal: Stereotypes (e.g., not fit into stereotypes, or fit into stereotypes which dismisses individual achievement)\\
\item
  Professional: Skepticism (e.g., microinvalidations when immigrant professionals' credentials and qualifications are challenged )
\end{itemize}

Sensemaking model by \citep{Weick_1995} explains how one can retrospectively make sense of past events and respond to future events.
CSM helps make sense of immigrant professional's experiences through the lenses of \textbf{power} (e.g., dominant-nondominant interactions).

To counter, immigrant professionals

\begin{itemize}
\item
  create another selves

  \begin{itemize}
  \tightlist
  \item
    muting/creating dual selves
  \item
    giving in
  \item
    giving up/ dissociating self
  \end{itemize}
\item
  rationalizes

  \begin{itemize}
  \tightlist
  \item
    perspective-taking
  \item
    blaming ignorance
  \item
    dismissing
  \item
    using humor
  \end{itemize}
\item
  takes ownership

  \begin{itemize}
  \tightlist
  \item
    normalizing
  \item
    appreciating cultural differences
  \item
    adapting to disparate expectation
  \end{itemize}
\end{itemize}

\citep{Williams_2017}

Communication among stakeholders in high reliability organizations (HROs)

organizational discourse: how members make sense of the tragedy by sharing.

critical team in high-hazard organization needs effective communication processes.\\
HROs are ``systems that successfully operate in environments that could produce catastrophic errors.''

3 broad themes from the grounded theory approach appear:

\begin{itemize}
\item
  Emotion

  \begin{itemize}
  \tightlist
  \item
    Some take time off to process the news.
  \item
    Some get back to work to cope with the events.
  \end{itemize}
\item
  Sensemaking (why)

  \begin{itemize}
  \item
    Debriefing process to understand what happened and learn from what happened.
  \item
    Purpose of sensemaking:

    \begin{itemize}
    \tightlist
    \item
      How could this have happened? This could happen to any other team. The fatal team was ``unlucky''.
    \item
      Why has this not happened to our team?
    \end{itemize}
  \end{itemize}
\item
  Learning (What now?)

  \begin{itemize}
  \tightlist
  \item
    Individual as well as organization(structural changes) can learn
  \end{itemize}
\end{itemize}

Making changes after a tragedy in the eyes of the crew was a routine event that officials make whenever a tragedy happens regardless.

Staying away from blame\\
Then the question is if you did find a person's fault led to the deaths of 19 people, we can you communicate that knowledge to facilitate learning.
Moreover, the attitude of the firefighters were reluctant to changes and went back to the basics, maybe because of this blameless culture.Hence, people might blame luck in this situation.
Interestingly, this blameless culture also facilitate group cohesion in the HROs.

A reconciliation is to recognize hindsight bias when trying to sensemaking/learning and avoid blaming.

\citep{Zanin_2019}

Athletes do not report concussion readily.
They often conceal it due to cultural discourses and norms.

\textbf{Cultural narratives}

Based on \citep{Polkinghorne_1995} two-level conceptualization of narrative: actors use narrative to create social reality and to make sense of their experiences.

Sport narrative and sensemaking:

\begin{itemize}
\item
  Sensemaking is the basis for social action.

  \begin{itemize}
  \tightlist
  \item
    Sensemaking is where meanings materialize to create identity.
  \end{itemize}
\item
  Cultural narratives help actors sensemaking by giving them a framework to understand an event.
\end{itemize}

Method: Abductive approach\\
Text Arichival Data: identify protagonist, actors, storyline, story values, and morals for each story.
Then, identify sport story archetypes

Interview {[}\url{Data://}{]}(\url{Data:/})\{.uri\} using constant comparative analysis to see how stakeholder made sense of a concussion even and reporting behavior, then compare to the types identified in the text archival data.

Findings:

5 narratives identified:

\begin{itemize}
\tightlist
\item
  Play-Through-Pain: enduring pain for the benefits of the team.
\item
  Big Leagues: American Dream of becoming a professional athlete through hard work and \textbf{perseverance.}
\item
  Commodification: abstract objects with financial value
\item
  Masculine Warrior: protagonist defeats an opponent through strength, toughness, bravery, violence, and \textbf{perseverance}
\item
  Need-for-Safety: Contemporary culture where ``athletes that seek healthcare are framed as moral and intelligent.''
\end{itemize}

Stakeholders refer to these 5 narratives to make sense of reporting behaviors.

Sensemaking use cultural sport narrative

\begin{enumerate}
\def\labelenumi{\arabic{enumi}.}
\tightlist
\item
  to extract cues: whether you have a concussion or not
\item
  construct identity: positive defense mechanism (4 over 5 narratives).
\end{enumerate}

\hypertarget{constitutive-communication-of-organizations}{%
\chapter{Constitutive Communication of Organizations}\label{constitutive-communication-of-organizations}}

\begin{itemize}
\tightlist
\item
  Social Constructionist
\item
  Structuration Theory: creation and reproduction of social systems that is based on the analysis of both structure and agents
\item
  little d discourse: what happen in the conversion (i.e., representation)
\item
  big D Discourse; The system of expectation you
\end{itemize}

\citep{Schoeneborn_2017}

6 premises:

\begin{itemize}
\tightlist
\item
  studies communication events (temporal and spatial dimensions).
\item
  Should be as inclusive as possible in its definition of (organizational) communication.
\item
  the co-constructed nature of (organizational) communication
\item
  who or what is acting is an open question
\item
  Communication events as unit of analysis
\item
  Equal importance of organizing (process) and organization (entity)
\end{itemize}

3 schools in CCO

\begin{itemize}
\item
  Montreal School approach: (pioneered by James R. Taylor) focus on text, speech, and linguistic forms to understand the their organizing properties.
  Organization is ``enacted through interaction and is related to processes of meaning negotiation''.

  \begin{itemize}
  \tightlist
  \item
    Cocretation: people talk \(\to\) interaction
  \item
    Distanciation: through time, separated, distanced from the original conversation.
  \item
    based on actor-network theory
  \end{itemize}
\item
  Four Flows approach (pioneered Robert D. McPhee): based on Giddens' structuration theory.
  Organization is created only when there are four flows:

  \begin{itemize}
  \tightlist
  \item
    Membership negotiation
  \item
    Self-structuring: constantly structuring, self here is the organization created through interaction.
  \item
    Activity coordination
  \item
    Institutional positioning (its environment)
  \end{itemize}
\item
  Social System Theory approach (pioneered by Niklas Luhmann): ``communication constitutes systems that produce the very elements they consist of, in a self-referential way''
\end{itemize}

Key Questions

\begin{itemize}
\tightlist
\item
  Ontological question: ``what is an organization?''
\item
  Composition problem: ``How to scale up from interaction to organization?''
\item
  Agency: ``Who ro what is able to act on behalf of the organization?''
\end{itemize}

Critiques:

\begin{itemize}
\tightlist
\item
  Bold claim that communication is organization
\item
  Too broad definition of communication.
\item
  Talk is cheap.
\end{itemize}

Emerging topics in CCO:

\begin{itemize}
\tightlist
\item
  Authority (power, domination, legitimization)
\item
  Disordering properties of communication.
\end{itemize}

\citep{Bruscella_2018}

Example of Four Flows school

terrorist organizations are communicatively constituted by the way they refer to their material as evidence to their image, existence, and legitimacy.

organization are constructed from the following communication processes:

\begin{itemize}
\tightlist
\item
  \textbf{Self-structuring}: division of labor, rights and responsibility
\item
  \textbf{membership negotiation}: membership inclusion and exclusion criteria.
\item
  \textbf{activity coordination}: mutual adjustments of action
\item
  \textbf{institutional position}: defining the boundaries of the organization
\end{itemize}

Question of agency:

\begin{itemize}
\tightlist
\item
  Four Flows theory define agency as human unique ability to make their own choice, while the Montreal School define agency as the ability to make a difference (e.g., humans or nonhumans). Hence, this paper included materials into the Four Flow Theory as the
\end{itemize}

Materials (or economics) can give inference about legitimacy, permanence, and credibility.

Hidden organization challenged the assumption of visibility from the Montreal school.

ISIL used a propaganda magazine (Dabiq) - communication- to illustrate their image and identity to its members

three communication strategies in their institutional positioning communication:

\begin{enumerate}
\def\labelenumi{(\arabic{enumi})}
\tightlist
\item
  instantiation: give artifacts to explain arguments.
\item
  cooptation: ``adoption of a rival's messaging for a purpose different from its original use.''
\item
  intertextual allusion: " a language form in which an association with a sacred, mythic, or origin text is insinuated by way of communication shortcuts."
\end{enumerate}

\citep{Koschmann_2015}

\citep{Knuf_1993} defines organizational rituals ``in terms of their formality, sacredness, irrationality, and aesthetics.''

``what rituals do is make present an authoritative text, and how they do this is through the attribution and appropriation of possessive constitution.''

Organizations is an ``abstract textual representations of power and legitimacy that are manifest in practice.'' Hence, Certain kinds of interactions (i.e., organizational rituals) create organization.

Authoritative text ``portrays the structure of the organization in ways that specify roles, duties, values, activities, outcomes, and the like, while also explaining relations of power and legitimacy''.

Specifically rituals found in this study:

\begin{itemize}
\tightlist
\item
  The opening
\item
  Sharing the critter: appreciation
\item
  Card signing
\item
  Spanish lesson
\item
  Reciting the mission statement
\item
  Moment of silence
\end{itemize}

Ritual Agency

\begin{itemize}
\tightlist
\item
  Rituals remind
\item
  Rituals discipline: instill or and constraining behavior.
\end{itemize}

Inclusion is authoritative text which constructs their organization.
Hence, rituals are practices that shows inclusion.

\citep{Cooren_2015}

ventriloquism denotes ``action through which someone or something makes someone or something else say or do things''.
\citep{Cooren_2010}.
For example, a layer is a ventriloquist while a contract is a dummy or figure.

\begin{itemize}
\tightlist
\item
  Ventriloquism is bi-directionality.
\item
  Figure or dummy can increase ventriloquist's authority
\item
  communication becomes the means through which some aspects of the world contradict or align themselves with other aspects of the world. From a ventriloqual point of view, the world is not a place where communication is detached from the things that matter."
\end{itemize}

\citep{Trittin_2015}

Diversity in organization cannot be superficially achieved by pre-defined unchanged characteristics (i.e., gender).
Hence, in this study, authors defined diversity as ``the plurality of''voices," that is, the range of individual opinions and societal discourses that get expressed and can find resonance in organizational settings."

\begin{itemize}
\tightlist
\item
  One can have many voices, and one voice can be manifested by multiple individuals.
\end{itemize}

Instrumental Perspectives on Diversity Management:

\begin{itemize}
\tightlist
\item
  Traditionally, diversity was thought as the difference between individuals, where communication (unidirectional, controllable, and linear process of information transmission) is a moderator of diversity on performance.
\item
  Later, diversity as diversified value orientation \citep{Eastman_2003}, work styles \citep{Shelton_2002}, education background \citep{Kearney_2009}.
\end{itemize}

Critical Perspectives on Diversity Management:

\begin{itemize}
\tightlist
\item
  Radical -critical: you can't manage diversity in organizational settings.
\item
  Constructive critical: instrumental approach can be both economically successful and socially just.
\end{itemize}

This paper follows the Montreal school of thought.

\hypertarget{socialization}{%
\chapter{Socialization}\label{socialization}}

\citep{Kramer_2011}

According to \citep{Van_Maanen_1979}, socialization is defined as ``the process by which an individual acquires the social knowledge and skills necessary to assume an organizational role.''

Levels of analysis:

\begin{itemize}
\item
  Single Organization Voluntary Socialization: Individual voluntary membership:

  \begin{itemize}
  \tightlist
  \item
    Membership negotiation
  \end{itemize}
\item
  How individuals' multiple group memberships interact to affect their socialization
\item
  how the multiple group memberships of other influence the socialization process of an individual.
\end{itemize}

Personally, I'd not define the way the author structured the research as levels of analysis because they are all at the individual level.

Personalization: new members try to change aspects of the organization to fit their needs.

Communication:

\begin{itemize}
\tightlist
\item
  Reconnaissance communication: is when prospective members to obtain info about the organization.
\end{itemize}

Membership statuses are fluid, and transitory, overlapping

\citep{Myers_2010}

Vocational Anticipatory Socialization (VAS) tries to predict individual's interests and their career pursuit using socialization theory.

Factors affecting the number of students choosing STEM field:

\begin{itemize}
\tightlist
\item
  Social factors
\item
  Personal interest
\end{itemize}

Sources of VAS:

\begin{itemize}
\tightlist
\item
  Family members: especially parents, socialize their children to various notions of jobs and careers.
\item
  Educational Institution: learn about power and social skills which later affects career choice.
\item
  Part-time jobs: good start for students to be socialize into the career network.
\item
  Peers: influence expectation of a future career.
\item
  Media: socialize value and expectations about careers.
\end{itemize}

Career Development Models :

\begin{enumerate}
\def\labelenumi{\arabic{enumi}.}
\tightlist
\item
  life-space model \citep{Vondracek_2019}:
\end{enumerate}

\begin{itemize}
\tightlist
\item
  Physiological factor (e.g., country of origin, genetics).
\item
  Psychological characteristics: (e.g., self-concept, development of intelligence, values, needs, interests, ability, aptitudes).
\item
  Socioeconomic environments
\end{itemize}

5 life stages:

\begin{itemize}
\tightlist
\item
  Growth (0-14)
\item
  Exploration (15-24)
\item
  Establishment (25-44)
\item
  Maintenance (45-64)
\item
  Decline (after 65)
\end{itemize}

9 roles:

\begin{itemize}
\tightlist
\item
  Child
\item
  Leisurite
\item
  Citizen
\item
  Worker
\item
  Pensioner
\item
  Spouse
\item
  Homemaker
\item
  Parent
\end{itemize}

\begin{enumerate}
\def\labelenumi{\arabic{enumi}.}
\setcounter{enumi}{1}
\tightlist
\item
  Social-cognitive career choice model: \citep{Lent_1994}
\end{enumerate}

Self-efficacy mediate the relationship between ability and interests.
A feedback loop is created once a person form career choice goals from self-efficacy and outcome expectations

VAS differs from these two models that it studies the socializing agents.

Found both gender (even though students deny such an effect, but they admit the social effects of others) and culture and Socioeconomic Status affect career choice.
\textbf{Experience} (exposure, job shadowing), \textbf{personal factors} (i.e., individual-level variables) also affect career choice (consistent with social-cognitive career).

VAS Messages:

\begin{itemize}
\tightlist
\item
  Value (e.g., family)
\item
  Expectation (e.g., self expectation of career)
\item
  Prescription (e.g., career choice should based on talents, interests,career's prestige and income potential )
\item
  Opportunity (e.g., take careers that are under pursued, hence more job opportunities).
\item
  Description (e.g., t job-specific environments, tasks, satisfaction, and required knowledge)
\end{itemize}

Check \citep[pp.107]{Myers_2010} for framework of VAS in STEM.

\citep{smith2012}:

Master narratives should be understood in tandem with personal narratives.
\citep[pp.209]{tannen2008} defines master narrative as ``a culture-wide ideology that shapes the big-N Narrative.'' In contrast with small-n where it personal stories and experiences can be found, Big-N Narratives are those that create a background for small-n narratives.

Retirement is a socialization process of the master narrative of aging and the American dream (e.g., success, and freedom - financial, responsibility).

Groups:

\begin{itemize}
\tightlist
\item
  Anticipatory group
\item
  Early work life
\item
  Preretirement
\item
  Retiree
\end{itemize}

Fractures of the master narrative:

\begin{itemize}
\tightlist
\item
  Freedom/routine fracture: they still want some work (structure), to stay active and productive members of society
\item
  Individual responsibility/universal expectations fracture: individual is responsible for one's happiness.
\end{itemize}

\citep{ferguson2017}

\citep{jablin_1987} defines socialization as a ``developmental unending process which can be broken up into three stages: \textbf{anticipatory, assimilation, and exit}.''

assimilation with others African American.
Later on, in college, the author tried closet his identity and desire.

\citep{Gibson_2000}

Organizational Assimilation Processes: blue-collar usually seen as routine and repetitive, tedious hence less creative, less motivation.

Consent that they need money and later on assimilate into the organizations.
Formally, \citep[pp.712]{jablin_1987} defines organizational assimilation as ``those ongoing behavioral and cognitive processes by which individuals join, become integrated into, and exit organizations.''

Stages of socialization:

\begin{itemize}
\tightlist
\item
  Anticipatory socialization
\item
  encounter
\item
  metamorphosis: accepted into the organization, and consistent with the organization's expectation. (outgroup \(\to\) ingroup)
\end{itemize}

Concertive control is ``a form of organizational control that emerges in accordance with the dominate ideologies in the organization, usually managerial-based.''

Workers construct hard-working identity and are proud of it.

\hypertarget{organizational-change}{%
\chapter{Organizational Change}\label{organizational-change}}

\citep{Lewis_2019} Organizations are ``socially constructed largely through the communicative interactions of internal and external stakeholders''.\\
Stakeholders are those ``who have a stake in an organization's process and or outputs''.\\
Ripple effects are ``the impacts that organizational actions and presence bring to stakeholders within and surrounding the organization''.

Even though the fad nature of society values change and associate with positive terms, compared to negative connotations of stability. However, changes does not equate good.\\
Triggers for organizational changes:\\
(external)

\begin{itemize}
\tightlist
\item
  Legal requirements\\
\item
  Stakeholders\\
\item
  Current business, societal, environmental trends\\
\item
  Technologies\\
\item
  Availability of financial resources\\
\item
  Alteration of relationship, powers, and global economy.
\end{itemize}

(internal)

\begin{itemize}
\tightlist
\item
  innovation
\item
  serendipity
\end{itemize}

Communication is key for changes because not until stakeholders recognize and communicate change that it materializes in an organization.

Sensemaking is both ``authoring'' and interpretation \citep{Lewis_2019}. Communication among stakeholders is at the hart of change processes in organizations because of this highly social process of making sense of what is going on and ``spinning it into narratives and theories of the world around us.'' \citep{Lewis_2019}.

Costs of change:

\begin{itemize}
\item
  Financial\\
\item
  Opportunities:

  \begin{itemize}
  \tightlist
  \item
    Lost productivity\\
  \item
    Lost time in training works\\
  \item
    Workflow
  \item
    Loss of high value stakeholders.\\
  \end{itemize}
\item
  Miscommunication: Confusion, fatigue.\\
\item
  Brand
\end{itemize}

\citep[p.10]{Zorn_1999} define organizational change as ``any alteration or modification of organizational structures or processes.''

Process of change:

\begin{itemize}
\tightlist
\item
  Innovation: (creating) idea generation\\
\item
  Adoption: (deciding) formal decision by leaders\\
\item
  Diffusion: (sharing) sharing of ideas.\\
\item
  Implementation: ``the translation of any tool or technique, process, or method of doing, from knowledge to practice.'' \citep{Tornatzky_1982}
\item
  Discontinuation: later, changes will become obsolete and a new cycle begins.
\end{itemize}

Communication is at the heart of all of these phases.

For relationships between innovation, diffusion, adoption, and implementation, check \citep[pp.~35]{Lewis_2019}

Types of Organizational change:

\begin{itemize}
\item
  Planned vs.~unplanned changes
\item
  Objects that are changed (e.g., technologies, programs, policies, processes, personal). But not good in practice due to blur lines among these objects.
\item
  Discursive change (i.e., new label for old things to fake change) vs.~material change (i.e., real changes in terms of operations, practices, relationships, decision-making) \citep[pp.10]{Zorn_1999}\\
\item
  Size and scope of change \citep{Bartunek_1987} (however, size and scope can be subjective):

  \begin{itemize}
  \tightlist
  \item
    First-order changes: small\\
  \item
    Second-order changes: large transformations, disruptive\\
  \item
    Third-order changes: continuous change.
  \end{itemize}
\end{itemize}

Combinations of these types of organizational change can be viewed in \citep[pp.~42]{Lewis_2019}

Complexity of change within organizations:

\begin{itemize}
\item
  Interdependence: " The degree to which stakeholders impact the lives of other stakeholders as they engage change." \citep{Lewis_2019}

  \begin{itemize}
  \tightlist
  \item
    Sequential Interdependence: Stakeholders affect one another in sequence (e.g., assembly line).\\
  \item
    Reciprocal interdependence: stakeholder's input are another stakeholder's outputs and vice versa. (e..g, co-authors).\\
  \end{itemize}
\item
  organizational structures:

  \begin{itemize}
  \item
    Structures: are rules and resources (e.g., information, status, organizational beliefs, ) that create organizational practices\\
  \item
    Types of Structures:

    \begin{itemize}
    \tightlist
    \item
      Decision-making patterns\\
    \item
      Decision0making processes\\
    \item
      Ladders of authority\\
    \item
      Role relationships\\
    \item
      Information-sharing norms\\
    \item
      Communication networks\\
    \item
      Reward system\\
    \end{itemize}
  \end{itemize}
\item
  Politics
\end{itemize}

Key processes in communication of planned change

\begin{itemize}
\tightlist
\item
  Dissemination of information\\
\item
  Soliciting input\\
\item
  Socialization
\end{itemize}

Types of communication in change implementation

\begin{itemize}
\tightlist
\item
  Formal Communication\\
\item
  Informal Communication: ``includes spontaneous interactions of stakeholders with each other, with implementers, and with non-stakeholders.''
\end{itemize}

Although we talked in this book extensively about change. However, changes are not always good. Sometimes, traditions are in place for a reason or reasons: If something works for a long time, it is likely to be robust. More on this idea can be read in Nassim Taleb's books.

\hypertarget{appendix-appendix}{%
\appendix}


  \bibliography{book.bib,packages.bib,references.bib}

\end{document}
